\documentclass[]{article}
\usepackage{lmodern}
\usepackage{amssymb,amsmath}
\usepackage{ifxetex,ifluatex}
\usepackage{fixltx2e} % provides \textsubscript
\ifnum 0\ifxetex 1\fi\ifluatex 1\fi=0 % if pdftex
  \usepackage[T1]{fontenc}
  \usepackage[utf8]{inputenc}
\else % if luatex or xelatex
  \ifxetex
    \usepackage{mathspec}
  \else
    \usepackage{fontspec}
  \fi
  \defaultfontfeatures{Ligatures=TeX,Scale=MatchLowercase}
\fi
% use upquote if available, for straight quotes in verbatim environments
\IfFileExists{upquote.sty}{\usepackage{upquote}}{}
% use microtype if available
\IfFileExists{microtype.sty}{%
\usepackage{microtype}
\UseMicrotypeSet[protrusion]{basicmath} % disable protrusion for tt fonts
}{}
\usepackage[margin=1in]{geometry}
\usepackage{hyperref}
\hypersetup{unicode=true,
            pdftitle={ISyE 6404 (EP-2): K-M Estimation, Kernel Regression and Spline},
            pdfauthor={Yuan Gao, Kevin Lee, Akshay Govindaraj; Yijun (Emma) Wan, Peter Williams, Ruixuan Zhang; ygao390, kylee20, ywan40, agovindaraj6, pwilliams60, rzhang438 \textbar{} @gatech.edu},
            pdfborder={0 0 0},
            breaklinks=true}
\urlstyle{same}  % don't use monospace font for urls
\usepackage{color}
\usepackage{fancyvrb}
\newcommand{\VerbBar}{|}
\newcommand{\VERB}{\Verb[commandchars=\\\{\}]}
\DefineVerbatimEnvironment{Highlighting}{Verbatim}{commandchars=\\\{\}}
% Add ',fontsize=\small' for more characters per line
\usepackage{framed}
\definecolor{shadecolor}{RGB}{248,248,248}
\newenvironment{Shaded}{\begin{snugshade}}{\end{snugshade}}
\newcommand{\KeywordTok}[1]{\textcolor[rgb]{0.13,0.29,0.53}{\textbf{#1}}}
\newcommand{\DataTypeTok}[1]{\textcolor[rgb]{0.13,0.29,0.53}{#1}}
\newcommand{\DecValTok}[1]{\textcolor[rgb]{0.00,0.00,0.81}{#1}}
\newcommand{\BaseNTok}[1]{\textcolor[rgb]{0.00,0.00,0.81}{#1}}
\newcommand{\FloatTok}[1]{\textcolor[rgb]{0.00,0.00,0.81}{#1}}
\newcommand{\ConstantTok}[1]{\textcolor[rgb]{0.00,0.00,0.00}{#1}}
\newcommand{\CharTok}[1]{\textcolor[rgb]{0.31,0.60,0.02}{#1}}
\newcommand{\SpecialCharTok}[1]{\textcolor[rgb]{0.00,0.00,0.00}{#1}}
\newcommand{\StringTok}[1]{\textcolor[rgb]{0.31,0.60,0.02}{#1}}
\newcommand{\VerbatimStringTok}[1]{\textcolor[rgb]{0.31,0.60,0.02}{#1}}
\newcommand{\SpecialStringTok}[1]{\textcolor[rgb]{0.31,0.60,0.02}{#1}}
\newcommand{\ImportTok}[1]{#1}
\newcommand{\CommentTok}[1]{\textcolor[rgb]{0.56,0.35,0.01}{\textit{#1}}}
\newcommand{\DocumentationTok}[1]{\textcolor[rgb]{0.56,0.35,0.01}{\textbf{\textit{#1}}}}
\newcommand{\AnnotationTok}[1]{\textcolor[rgb]{0.56,0.35,0.01}{\textbf{\textit{#1}}}}
\newcommand{\CommentVarTok}[1]{\textcolor[rgb]{0.56,0.35,0.01}{\textbf{\textit{#1}}}}
\newcommand{\OtherTok}[1]{\textcolor[rgb]{0.56,0.35,0.01}{#1}}
\newcommand{\FunctionTok}[1]{\textcolor[rgb]{0.00,0.00,0.00}{#1}}
\newcommand{\VariableTok}[1]{\textcolor[rgb]{0.00,0.00,0.00}{#1}}
\newcommand{\ControlFlowTok}[1]{\textcolor[rgb]{0.13,0.29,0.53}{\textbf{#1}}}
\newcommand{\OperatorTok}[1]{\textcolor[rgb]{0.81,0.36,0.00}{\textbf{#1}}}
\newcommand{\BuiltInTok}[1]{#1}
\newcommand{\ExtensionTok}[1]{#1}
\newcommand{\PreprocessorTok}[1]{\textcolor[rgb]{0.56,0.35,0.01}{\textit{#1}}}
\newcommand{\AttributeTok}[1]{\textcolor[rgb]{0.77,0.63,0.00}{#1}}
\newcommand{\RegionMarkerTok}[1]{#1}
\newcommand{\InformationTok}[1]{\textcolor[rgb]{0.56,0.35,0.01}{\textbf{\textit{#1}}}}
\newcommand{\WarningTok}[1]{\textcolor[rgb]{0.56,0.35,0.01}{\textbf{\textit{#1}}}}
\newcommand{\AlertTok}[1]{\textcolor[rgb]{0.94,0.16,0.16}{#1}}
\newcommand{\ErrorTok}[1]{\textcolor[rgb]{0.64,0.00,0.00}{\textbf{#1}}}
\newcommand{\NormalTok}[1]{#1}
\usepackage{longtable,booktabs}
\usepackage{graphicx,grffile}
\makeatletter
\def\maxwidth{\ifdim\Gin@nat@width>\linewidth\linewidth\else\Gin@nat@width\fi}
\def\maxheight{\ifdim\Gin@nat@height>\textheight\textheight\else\Gin@nat@height\fi}
\makeatother
% Scale images if necessary, so that they will not overflow the page
% margins by default, and it is still possible to overwrite the defaults
% using explicit options in \includegraphics[width, height, ...]{}
\setkeys{Gin}{width=\maxwidth,height=\maxheight,keepaspectratio}
\IfFileExists{parskip.sty}{%
\usepackage{parskip}
}{% else
\setlength{\parindent}{0pt}
\setlength{\parskip}{6pt plus 2pt minus 1pt}
}
\setlength{\emergencystretch}{3em}  % prevent overfull lines
\providecommand{\tightlist}{%
  \setlength{\itemsep}{0pt}\setlength{\parskip}{0pt}}
\setcounter{secnumdepth}{0}
% Redefines (sub)paragraphs to behave more like sections
\ifx\paragraph\undefined\else
\let\oldparagraph\paragraph
\renewcommand{\paragraph}[1]{\oldparagraph{#1}\mbox{}}
\fi
\ifx\subparagraph\undefined\else
\let\oldsubparagraph\subparagraph
\renewcommand{\subparagraph}[1]{\oldsubparagraph{#1}\mbox{}}
\fi

%%% Use protect on footnotes to avoid problems with footnotes in titles
\let\rmarkdownfootnote\footnote%
\def\footnote{\protect\rmarkdownfootnote}

%%% Change title format to be more compact
\usepackage{titling}

% Create subtitle command for use in maketitle
\newcommand{\subtitle}[1]{
  \posttitle{
    \begin{center}\large#1\end{center}
    }
}

\setlength{\droptitle}{-2em}

  \title{ISyE 6404 (EP-2): K-M Estimation, Kernel Regression and Spline}
    \pretitle{\vspace{\droptitle}\centering\huge}
  \posttitle{\par}
    \author{Yuan Gao, Kevin Lee, Akshay Govindaraj \\ Yijun (Emma) Wan, Peter Williams, Ruixuan Zhang \\ ygao390, kylee20, ywan40, agovindaraj6, pwilliams60, rzhang438
\textbar{} @gatech.edu}
    \preauthor{\centering\large\emph}
  \postauthor{\par}
      \predate{\centering\large\emph}
  \postdate{\par}
    \date{2018-10-22}


\begin{document}
\maketitle

{
\setcounter{tocdepth}{2}
\tableofcontents
}
\subsection{1. K-M Estimation (25\%):}\label{k-m-estimation-25}

\emph{Locate a data set with right-censoring (in Type-I Censoring) in
the field of your interest, e.g., eCommerce, medical study, drug
development, supply-chain/logistics operations, for applying the K-M
Estimator to estimate the survival function with pointwise confidence
intervals.}

For this exercise, we located a dataset, that consists of measures on 69
different patients who received a heart transplant, taken from the first
edition of the text \emph{The Statistical Analysis of Failure Time Data}
by Kalbfleisch and Prentice, Appendix I (230-232), published originally
in 1980, which can also be found via the following link on the Carnegie
Mellon statistics site: \url{http://lib.stat.cmu.edu/datasets/stanford},
and has three columns with the following measures:

\begin{itemize}
\tightlist
\item
  Age: Age of patient in years at the time of heart transplant in years
\item
  Status: Survival status (1=dead, 0=alive)
\item
  Days: Survival time in days after transplant (in days)
\end{itemize}

\begin{longtable}[]{@{}rrr@{}}
\caption{Preview: Heart Transplant Data}\tabularnewline
\toprule
Age & Status & Days\tabularnewline
\midrule
\endfirsthead
\toprule
Age & Status & Days\tabularnewline
\midrule
\endhead
41 & 1 & 5\tabularnewline
40 & 1 & 16\tabularnewline
35 & 0 & 39\tabularnewline
50 & 1 & 53\tabularnewline
45 & 1 & 68\tabularnewline
26 & 0 & 180\tabularnewline
\bottomrule
\end{longtable}

This dataset is right censored, because as shown above, we don't exactly
how long patients who currently have a (status=0) will survive, or
survived.

Utilizing this dataset, and the \emph{survival} package in R, we
generate the Kaplan-Meier estimates, and visualize the survival
function, i.e. \(S_{KM}(x_{i:n}) = 1 - F_{KM}(x_{i:n})\), with
confidence intervals below. Note that the R code utilized is also shown:

\newpage

\subparagraph{R Code}\label{r-code}

\begin{Shaded}
\begin{Highlighting}[]
\NormalTok{km_fit <-}\StringTok{ }\KeywordTok{survfit}\NormalTok{(}\KeywordTok{Surv}\NormalTok{(}\DataTypeTok{time =}\NormalTok{ Days, }\DataTypeTok{event =}\NormalTok{ Status) }\OperatorTok{~}\StringTok{ }\DecValTok{1}\NormalTok{, }\DataTypeTok{data =}\NormalTok{ hdat, }
                  \DataTypeTok{type =} \StringTok{'kaplan-meier'}\NormalTok{) }\CommentTok{#only one group of patients}
\NormalTok{km_dat <-}\StringTok{ }\KeywordTok{data.frame}\NormalTok{(}\DataTypeTok{time=}\NormalTok{km_fit}\OperatorTok{$}\NormalTok{time,}
                     \DataTypeTok{survival=}\NormalTok{km_fit}\OperatorTok{$}\NormalTok{surv,}
                     \DataTypeTok{upper_ci =}\NormalTok{ km_fit}\OperatorTok{$}\NormalTok{upper, }
                     \DataTypeTok{lower_ci =}\NormalTok{ km_fit}\OperatorTok{$}\NormalTok{lower) }\CommentTok{#model results}
\NormalTok{melted_km_dat <-}\StringTok{ }\KeywordTok{melt}\NormalTok{(km_dat, }\DataTypeTok{id.vars =} \KeywordTok{c}\NormalTok{(}\StringTok{'time'}\NormalTok{)) }\CommentTok{#transform for viz}
\KeywordTok{ggplot}\NormalTok{(}\KeywordTok{aes}\NormalTok{(}\DataTypeTok{x=}\NormalTok{time,}\DataTypeTok{y=}\NormalTok{value, }\DataTypeTok{color =}\NormalTok{ variable ,}\DataTypeTok{group =}\NormalTok{ variable), }\DataTypeTok{data=}\NormalTok{melted_km_dat) }\OperatorTok{+}
\StringTok{  }\KeywordTok{geom_step}\NormalTok{(}\DataTypeTok{size =} \FloatTok{1.25}\NormalTok{) }\OperatorTok{+}\StringTok{ }\KeywordTok{theme_bw}\NormalTok{() }\OperatorTok{+}\StringTok{ }\KeywordTok{xlab}\NormalTok{(}\StringTok{'Time (days)'}\NormalTok{) }\OperatorTok{+}\StringTok{ }
\StringTok{  }\KeywordTok{ylab}\NormalTok{(}\StringTok{'Survival Function'}\NormalTok{) }\OperatorTok{+}\StringTok{ }
\StringTok{  }\KeywordTok{ggtitle}\NormalTok{(}\StringTok{'K-M Estimates with 95% Confidence Bounds: Heart Transplant Patients'}\NormalTok{) }\OperatorTok{+}\StringTok{ }
\StringTok{  }\KeywordTok{labs}\NormalTok{(}\DataTypeTok{color=}\StringTok{'Curve'}\NormalTok{)}
\end{Highlighting}
\end{Shaded}

\begin{center}\includegraphics{enrichment_project2_fall18_files/figure-latex/km_estimate-1} \end{center}

Visual analysis of the plot here, indicates that the probability of
survival for 500 days, after a heart transplant is approximately 42\%,
with a 95\% confidence range of approximately 30-55\%, among patients in
this data, when this data was collected decades ago. We hope that
survival rates have increased significantly since this study was
conducted.

\subsection{2. Kernel and Related Regression with One Explanatory
Variable
(40\%):}\label{kernel-and-related-regression-with-one-explanatory-variable-40}

\emph{Locate a data set suitable for nonparametric regression (usually
has nonlinear y-x relationship) in the field of your interest, e.g.,
eCommerce, medical study, drug development, supply-chain/logistics
operations. Apply all of the procedures below:}

\begin{enumerate}
\def\labelenumi{\arabic{enumi})}
\tightlist
\item
  \emph{Kernel Regression,}\\
\item
  \emph{Local Polynomial Regression,}\\
\item
  \emph{LOESS,}\\
\item
  \emph{Smoothing Spline, to the y-x data-fit.}

  \begin{itemize}
  \tightlist
  \item
    Compare fits from the four methods.\\
    For this exercise, we located another dataset, that consists of Data
    give the mortality rate and the education level for 58 U.S. cities.
    It can be found via the following link on the \emph{The Data And
    Story Library}
    site:\url{https://dasl.datadescription.com/datafile/education-and-mortality},
    and has two columns with the following measures:
  \end{itemize}
\end{enumerate}

\begin{itemize}
\tightlist
\item
  Mortality: the mortality rate (deaths per 100,000 people)
\item
  Education: the average number of years in school
\end{itemize}

\begin{enumerate}
\def\labelenumi{\arabic{enumi})}
\tightlist
\item
  \emph{Kernel Regression,}
\end{enumerate}

\begin{Shaded}
\begin{Highlighting}[]
\NormalTok{rdat <-}\StringTok{ }\KeywordTok{read.csv}\NormalTok{(}\StringTok{'data/education-and-mortality.csv'}\NormalTok{,}\DataTypeTok{header=}\NormalTok{T,}\DataTypeTok{stringsAsFactors=}\NormalTok{F)}
\NormalTok{bw <-}\StringTok{ }\KeywordTok{npregbw}\NormalTok{(}\DataTypeTok{formula=}\NormalTok{rdat}\OperatorTok{$}\NormalTok{Mortality}\OperatorTok{~}\NormalTok{rdat}\OperatorTok{$}\NormalTok{Education, }\DataTypeTok{tol=}\NormalTok{.}\DecValTok{1}\NormalTok{, }\DataTypeTok{ftol=}\NormalTok{.}\DecValTok{1}\NormalTok{)}
\end{Highlighting}
\end{Shaded}

\begin{verbatim}
## 
Multistart 1 of 1 |
Multistart 1 of 1 |
Multistart 1 of 1 |
Multistart 1 of 1 |
Multistart 1 of 1 |
                   
\end{verbatim}

\begin{Shaded}
\begin{Highlighting}[]
\CommentTok{# Now take these bandwidths and fit the model and gradients}
\NormalTok{model <-}\StringTok{ }\KeywordTok{npreg}\NormalTok{(}\DataTypeTok{bws =}\NormalTok{ bw, }\DataTypeTok{gradients =} \OtherTok{TRUE}\NormalTok{)}
\KeywordTok{summary}\NormalTok{(model)}
\end{Highlighting}
\end{Shaded}

\begin{verbatim}
## 
## Regression Data: 58 training points, in 1 variable(s)
##               rdat$Education
## Bandwidth(s):      0.4382052
## 
## Kernel Regression Estimator: Local-Constant
## Bandwidth Type: Fixed
## Residual standard error: 47.22676
## R-squared: 0.4196534
## 
## Continuous Kernel Type: Second-Order Gaussian
## No. Continuous Explanatory Vars.: 1
\end{verbatim}

\begin{Shaded}
\begin{Highlighting}[]
\KeywordTok{npplot}\NormalTok{(}\DataTypeTok{bws=}\NormalTok{bw, }\DataTypeTok{plot.errors.method=}\StringTok{"bootstrap"}\NormalTok{,}\DataTypeTok{col =} \StringTok{"blue"}\NormalTok{)}
\KeywordTok{points}\NormalTok{(rdat}\OperatorTok{$}\NormalTok{Education, rdat}\OperatorTok{$}\NormalTok{Mortality, }\DataTypeTok{cex=}\NormalTok{.}\DecValTok{2}\NormalTok{, }\DataTypeTok{col=}\StringTok{"red"}\NormalTok{)}
\end{Highlighting}
\end{Shaded}

\begin{center}\includegraphics{enrichment_project2_fall18_files/figure-latex/Kernel Regression-1} \end{center}

\begin{enumerate}
\def\labelenumi{\arabic{enumi})}
\setcounter{enumi}{1}
\tightlist
\item
  \emph{Local Polynomial Regression,}
\end{enumerate}

\begin{Shaded}
\begin{Highlighting}[]
\NormalTok{fit <-}\StringTok{ }\KeywordTok{locfit}\NormalTok{(Mortality}\OperatorTok{~}\KeywordTok{lp}\NormalTok{(Education,}\DataTypeTok{deg=}\DecValTok{4}\NormalTok{),}\DataTypeTok{data=}\NormalTok{rdat)}
\KeywordTok{summary}\NormalTok{(fit)}
\end{Highlighting}
\end{Shaded}

\begin{verbatim}
## Estimation type: Local Regression 
## 
## Call:
## locfit(formula = Mortality ~ lp(Education, deg = 4), data = rdat)
## 
## Number of data points:  58 
## Independent variables:  Education 
## Evaluation structure: Rectangular Tree 
## Number of evaluation points:  5 
## Degree of fit:  4 
## Fitted Degrees of Freedom:  6.982
\end{verbatim}

\begin{Shaded}
\begin{Highlighting}[]
\KeywordTok{crit}\NormalTok{(fit) <-}\StringTok{ }\KeywordTok{crit}\NormalTok{(fit,}\DataTypeTok{cov=}\FloatTok{0.99}\NormalTok{)}
\KeywordTok{plot}\NormalTok{(fit,}\DataTypeTok{band=}\StringTok{'local'}\NormalTok{)}
\KeywordTok{points}\NormalTok{(rdat}\OperatorTok{$}\NormalTok{Education, rdat}\OperatorTok{$}\NormalTok{Mortality, }\DataTypeTok{cex=}\NormalTok{.}\DecValTok{2}\NormalTok{, }\DataTypeTok{col=}\StringTok{"red"}\NormalTok{)}
\end{Highlighting}
\end{Shaded}

\begin{center}\includegraphics{enrichment_project2_fall18_files/figure-latex/Local Polynomial Regression-1} \end{center}

\begin{Shaded}
\begin{Highlighting}[]
\NormalTok{res <-}\StringTok{ }\KeywordTok{residuals}\NormalTok{(fit)}
\NormalTok{des <-}\StringTok{ }\NormalTok{rdat}\OperatorTok{$}\NormalTok{Mortality}\OperatorTok{-}\KeywordTok{mean}\NormalTok{(rdat}\OperatorTok{$}\NormalTok{Mortality)}
\NormalTok{r_square =}\StringTok{ }\DecValTok{1}\OperatorTok{-}\StringTok{ }\KeywordTok{sum}\NormalTok{((res)}\OperatorTok{^}\DecValTok{2}\OperatorTok{/}\KeywordTok{sum}\NormalTok{((des)}\OperatorTok{^}\DecValTok{2}\NormalTok{))}
\end{Highlighting}
\end{Shaded}

\begin{enumerate}
\def\labelenumi{\arabic{enumi})}
\setcounter{enumi}{2}
\tightlist
\item
  \emph{LOESS,}
\end{enumerate}

\begin{Shaded}
\begin{Highlighting}[]
\CommentTok{#Making our variables easy to use in modeling}
\NormalTok{Mortality=rdat[,}\DecValTok{1}\NormalTok{]}
\NormalTok{Education=rdat[,}\DecValTok{2}\NormalTok{]}

\CommentTok{#packages to help create colors for plotting smoothing lines}
\KeywordTok{library}\NormalTok{(RColorBrewer)}
\KeywordTok{library}\NormalTok{(sp)}

\CommentTok{#Install "stats" package to be able to perform loess and spline data smoothing.}
\CommentTok{#install.packages('stats', dependencies=TRUE, repos='http://cran.rstudio.com/')}
\end{Highlighting}
\end{Shaded}

\section{Loess models, with varying spans, with degree=1. When
interpreting the summary
results}\label{loess-models-with-varying-spans-with-degree1.-when-interpreting-the-summary-results}

\section{for the loess models, we will notice that increasing span will
decrease the
SSR}\label{for-the-loess-models-we-will-notice-that-increasing-span-will-decrease-the-ssr}

\section{but the curve of the model will become more jagged, making it
so that the curve may
not}\label{but-the-curve-of-the-model-will-become-more-jagged-making-it-so-that-the-curve-may-not}

\section{be the best representation. Loess is mostly a visual tool to
see curves for non-linear curves
through}\label{be-the-best-representation.-loess-is-mostly-a-visual-tool-to-see-curves-for-non-linear-curves-through}

\section{the use of data to form the line rather than a specified
mathematical
model.}\label{the-use-of-data-to-form-the-line-rather-than-a-specified-mathematical-model.}

\section{Loess is based purely on the span and degree of the polynomial
for the potential loess
model.}\label{loess-is-based-purely-on-the-span-and-degree-of-the-polynomial-for-the-potential-loess-model.}

\begin{Shaded}
\begin{Highlighting}[]
\NormalTok{loess.model.}\DecValTok{1}\NormalTok{=}\KeywordTok{loess}\NormalTok{(Mortality}\OperatorTok{~}\NormalTok{Education,}\DataTypeTok{span=}\NormalTok{.}\DecValTok{25}\NormalTok{,}\DataTypeTok{degree=}\DecValTok{1}\NormalTok{)}
\NormalTok{loess.model.}\DecValTok{2}\NormalTok{=}\KeywordTok{loess}\NormalTok{(Mortality}\OperatorTok{~}\NormalTok{Education,}\DataTypeTok{span=}\NormalTok{.}\DecValTok{5}\NormalTok{,}\DataTypeTok{degree=}\DecValTok{1}\NormalTok{)}
\NormalTok{loess.model.}\DecValTok{3}\NormalTok{=}\KeywordTok{loess}\NormalTok{(Mortality}\OperatorTok{~}\NormalTok{Education,}\DataTypeTok{span=}\NormalTok{.}\DecValTok{7}\NormalTok{,}\DataTypeTok{degree=}\DecValTok{1}\NormalTok{)}
\NormalTok{loess.model.}\DecValTok{4}\NormalTok{=}\KeywordTok{loess}\NormalTok{(Mortality}\OperatorTok{~}\NormalTok{Education,}\DataTypeTok{span=}\NormalTok{.}\DecValTok{9}\NormalTok{,}\DataTypeTok{degree=}\DecValTok{1}\NormalTok{)}
\KeywordTok{summary}\NormalTok{(loess.model.}\DecValTok{1}\NormalTok{)}
\end{Highlighting}
\end{Shaded}

\begin{verbatim}
## Call:
## loess(formula = Mortality ~ Education, span = 0.25, degree = 1)
## 
## Number of Observations: 58 
## Equivalent Number of Parameters: 7.85 
## Residual Standard Error: 49.26 
## Trace of smoother matrix: 9.33  (exact)
## 
## Control settings:
##   span     :  0.25 
##   degree   :  1 
##   family   :  gaussian
##   surface  :  interpolate      cell = 0.2
##   normalize:  TRUE
##  parametric:  FALSE
## drop.square:  FALSE
\end{verbatim}

\begin{Shaded}
\begin{Highlighting}[]
\KeywordTok{summary}\NormalTok{(loess.model.}\DecValTok{2}\NormalTok{)}
\end{Highlighting}
\end{Shaded}

\begin{verbatim}
## Call:
## loess(formula = Mortality ~ Education, span = 0.5, degree = 1)
## 
## Number of Observations: 58 
## Equivalent Number of Parameters: 3.85 
## Residual Standard Error: 49.02 
## Trace of smoother matrix: 4.51  (exact)
## 
## Control settings:
##   span     :  0.5 
##   degree   :  1 
##   family   :  gaussian
##   surface  :  interpolate      cell = 0.2
##   normalize:  TRUE
##  parametric:  FALSE
## drop.square:  FALSE
\end{verbatim}

\begin{Shaded}
\begin{Highlighting}[]
\KeywordTok{summary}\NormalTok{(loess.model.}\DecValTok{3}\NormalTok{)}
\end{Highlighting}
\end{Shaded}

\begin{verbatim}
## Call:
## loess(formula = Mortality ~ Education, span = 0.7, degree = 1)
## 
## Number of Observations: 58 
## Equivalent Number of Parameters: 3.12 
## Residual Standard Error: 48.58 
## Trace of smoother matrix: 3.61  (exact)
## 
## Control settings:
##   span     :  0.7 
##   degree   :  1 
##   family   :  gaussian
##   surface  :  interpolate      cell = 0.2
##   normalize:  TRUE
##  parametric:  FALSE
## drop.square:  FALSE
\end{verbatim}

\begin{Shaded}
\begin{Highlighting}[]
\KeywordTok{summary}\NormalTok{(loess.model.}\DecValTok{4}\NormalTok{)}
\end{Highlighting}
\end{Shaded}

\begin{verbatim}
## Call:
## loess(formula = Mortality ~ Education, span = 0.9, degree = 1)
## 
## Number of Observations: 58 
## Equivalent Number of Parameters: 2.49 
## Residual Standard Error: 48.21 
## Trace of smoother matrix: 2.8  (exact)
## 
## Control settings:
##   span     :  0.9 
##   degree   :  1 
##   family   :  gaussian
##   surface  :  interpolate      cell = 0.2
##   normalize:  TRUE
##  parametric:  FALSE
## drop.square:  FALSE
\end{verbatim}

We will now plot the data points from our our mortality data,

\begin{Shaded}
\begin{Highlighting}[]
\KeywordTok{plot}\NormalTok{(Mortality}\OperatorTok{~}\NormalTok{Education)}
\end{Highlighting}
\end{Shaded}

\begin{center}\includegraphics{enrichment_project2_fall18_files/figure-latex/PlotofMortality-1} \end{center}

\begin{Shaded}
\begin{Highlighting}[]
\CommentTok{#The loess.smooth() function allows us to make fitted lines for the smoothed lines with varying spans}
\CommentTok{#We will plot various functions to be able to see the effect of span on the smoothing line}
\NormalTok{loessfit1=}\KeywordTok{loess.smooth}\NormalTok{(}\DataTypeTok{x=}\NormalTok{Education,}\DataTypeTok{y=}\NormalTok{Mortality,}\DataTypeTok{span=}\NormalTok{.}\DecValTok{25}\NormalTok{,}\DataTypeTok{degree=}\DecValTok{1}\NormalTok{)}
\NormalTok{loessfit2=}\KeywordTok{loess.smooth}\NormalTok{(}\DataTypeTok{x=}\NormalTok{Education,}\DataTypeTok{y=}\NormalTok{Mortality,}\DataTypeTok{span=}\NormalTok{.}\DecValTok{4}\NormalTok{,}\DataTypeTok{degree=}\DecValTok{1}\NormalTok{)}
\NormalTok{loessfit3=}\KeywordTok{loess.smooth}\NormalTok{(}\DataTypeTok{x=}\NormalTok{Education,}\DataTypeTok{y=}\NormalTok{Mortality,}\DataTypeTok{span=}\NormalTok{.}\DecValTok{6}\NormalTok{,}\DataTypeTok{degree=}\DecValTok{1}\NormalTok{)}
\NormalTok{loessfit4=}\KeywordTok{loess.smooth}\NormalTok{(}\DataTypeTok{x=}\NormalTok{Education,}\DataTypeTok{y=}\NormalTok{Mortality,}\DataTypeTok{span=}\NormalTok{.}\DecValTok{9}\NormalTok{,}\DataTypeTok{degree=}\DecValTok{1}\NormalTok{)}

\CommentTok{#Recall our plot with the mortality and education data. *****Needs legend*****}
\CommentTok{#We will now display the loess fits of varying span onto the plot to see the possible smoothing lines we can make with the loess #function.}
\KeywordTok{plot}\NormalTok{(Mortality}\OperatorTok{~}\NormalTok{Education)}
\KeywordTok{lines}\NormalTok{(loessfit1}\OperatorTok{$}\NormalTok{x,loessfit1}\OperatorTok{$}\NormalTok{y,}\DataTypeTok{col=}\StringTok{"blue"}\NormalTok{, }\DataTypeTok{lwd=}\DecValTok{2}\NormalTok{)}
\KeywordTok{lines}\NormalTok{(loessfit2}\OperatorTok{$}\NormalTok{x,loessfit2}\OperatorTok{$}\NormalTok{y,}\DataTypeTok{col=}\StringTok{"green"}\NormalTok{, }\DataTypeTok{lwd=}\DecValTok{2}\NormalTok{)}
\KeywordTok{lines}\NormalTok{(loessfit3}\OperatorTok{$}\NormalTok{x,loessfit3}\OperatorTok{$}\NormalTok{y,}\DataTypeTok{col=}\StringTok{"yellow"}\NormalTok{, }\DataTypeTok{lwd=}\DecValTok{2}\NormalTok{)}
\KeywordTok{lines}\NormalTok{(loessfit4}\OperatorTok{$}\NormalTok{x,loessfit4}\OperatorTok{$}\NormalTok{y,}\DataTypeTok{col=}\StringTok{"red"}\NormalTok{, }\DataTypeTok{lwd=}\DecValTok{2}\NormalTok{)}
\end{Highlighting}
\end{Shaded}

\begin{center}\includegraphics{enrichment_project2_fall18_files/figure-latex/Loess Smoothing-1} \end{center}

If we use 0.25 span for the case that it is the least jagged curve fit,
while still holding some obvious grouping of values visually. One thing
to keep note of is that the loess fit will be useful in visually
displaying the relationship between the two variables mortality and
education, but since our sample size is not particularly large, as loess
fitting requires quite a large sample size to utilize effectively.

\begin{Shaded}
\begin{Highlighting}[]
\CommentTok{#Smoothing Spline Plot}
\KeywordTok{plot}\NormalTok{(Mortality}\OperatorTok{~}\NormalTok{Education)}
\NormalTok{spline.mortality=}\KeywordTok{smooth.spline}\NormalTok{(Education,Mortality)}
\KeywordTok{lines}\NormalTok{(spline.mortality, }\DataTypeTok{col=}\StringTok{"purple"}\NormalTok{, }\DataTypeTok{lwd=}\DecValTok{2}\NormalTok{)}
\end{Highlighting}
\end{Shaded}

\begin{center}\includegraphics{enrichment_project2_fall18_files/figure-latex/Smoothing Spline-1} \end{center}

We can see that using a smoothing spline gives us a relatively linear
line with small bumps along it.

\begin{enumerate}
\def\labelenumi{\arabic{enumi})}
\setcounter{enumi}{3}
\tightlist
\item
  \emph{Smoothing Spline, to the y-x data-fit.}
\end{enumerate}

\begin{verbatim}
#Smoothing Spline Plot using the smooth.spline() function in {stats} package
plot(Mortality~Education)
spline.mortality=smooth.spline(Education,Mortality)
lines(spline.mortality, col="purple", lwd=2)
\end{verbatim}

We can see that using a smoothing spline gives us a relatively linear
line with small bumps along it.

Comparison for above Kernel Estimators: 1) the Nadaraya-Watson kernel
estimator suffers from bias, both at the boundaries due to the one-sided
neighborhoods and in the interior when the xi's are not uniformly
distributed. The reason is that in local constant modeling more or less
the same points are used to estimate the curve near the boundary. 2)
Local polynomial regression overcomes the above problem by fitting a
higher degree polynomial here. What's more, Local polynomial regression
also has other advantage such as,independence of the regression design
(fixed or random, highly clustered or nearly uniform), and efficiency in an
asymptotic minimax sense.

Comparison for Kernel Estimators and Nearest neighbor methods: 1) In
general, nearest neighbor and kernel estimators produce similar results.
For example: The loess approach is similar to the Nadaraya-Watson
approach in that both are taking linear combinations of the responses
\(y_i\)

\begin{enumerate}
\def\labelenumi{\arabic{enumi})}
\setcounter{enumi}{1}
\item
  In terms of bias and variance, the nearest neighbor estimator performs
  well if the variance decreases more than the squared bias increases.
  For example: Suppose that f is continuously differentiable up to
  second order and that \(K(x,x_0) = 0\) if \(|x − x_0| > h\) Then
  Loess: \[Bias{f(x_0) = O(h^2)}\] Nadaraya-Watson:
  \[Bias{f(x_0)} = f_0(x_0)\sum_{i}wi(x_i − x_0) + O(h^2),\] where
  \(w_i = K_h(x_i,x_0)/\sum_{j}K_h(x_j,x_0)\) The leading term for the
  bias of the Nadaraya-Watson estimator is referred to as design bias
  which is not present for loess estimators. In other words, loess
  performs naturally eliminates design bias, and the resulting estimator
  is free of bias up to second order.
\item
  Nearest neighbor methods is better if we have spaces with clustered
  design points followed by intervals with sparse design points.
\end{enumerate}

Comparison for Kernel Estimators, Nearest neighbor methods and spline
regressions:

\begin{enumerate}
\def\labelenumi{\arabic{enumi})}
\tightlist
\item
  In general, for one input variable and one output variable, smoothing
  splines can basically do everything which kernel regression can do.
\end{enumerate}

2)While a higher order polynomial will provide a closer fit at any
particular point, the loss of parsimony is not the only potential
problem of over fitting, unwanted oscillations can appear between data
points.Spline functions avoid this pitfall.

\begin{enumerate}
\def\labelenumi{\arabic{enumi})}
\setcounter{enumi}{2}
\tightlist
\item
  Kernels however are easier to program, analyze mathematically,extend
  more straightforwardly to multiple variables, and to combinations of
  discrete and continuous variables.
\end{enumerate}

4)Splines has faster computational speed and simplicity,as well as the
clarity of controlling curvature directly.

\subsection{\texorpdfstring{3. Cross-Validation With the
``Leave-One-Out'' Procedure
(10\%):}{3. Cross-Validation With the Leave-One-Out Procedure (10\%):}}\label{cross-validation-with-the-leave-one-out-procedure-10}

\emph{Compare the above four methods with a leave-one-out
cross-validation procedure.}

\subsection{4. Resampling Procedures: Bootstrap and Jackknife
(25\%):}\label{resampling-procedures-bootstrap-and-jackknife-25}

\begin{enumerate}
\def\labelenumi{\arabic{enumi})}
\tightlist
\item
  \emph{Select an input \(x_0\) in the {[}min(x-data),
  max(x-data){]}.}\\
\item
  \emph{Use all four regression models built in Task \#2 to make
  point-predictions of \(Y\) at \(x_0\).}\\
\item
  \emph{Use both bootstrap (\(B = 1000\)) and jackknife resampling
  procedures to find a 90\% pointwise confidence interval (CI) for the
  point-prediction. If the resampled distribution of the
  point-prediction is symmetric, use 5\% in each tail to find the
  CI-bounds. If the distribution is not symmetric, use the HPD-interval
  idea to find the CI-bound. Compare the results from four regression
  methods.}
\end{enumerate}


\end{document}
